%----------------------------------------------------------------------------
%----------------------------------------------------------------------------
%				    	SETUP
%----------------------------------------------------------------------------
%----------------------------------------------------------------------------

\documentclass[11pt]{article}

%----------------------------------------------------------------------------
%			  	   PACKAGES
%----------------------------------------------------------------------------

%%%%%%%%%%%%%%%%%%%%%%%
% 	  Packages
%%%%%%%%%%%%%%%%%%%%%%%

%% Fonts and Symbols
%% --------------------------
\usepackage{
	amsmath,			% math operators
	amssymb,			% math symbols
%	amsthm,				% theorem environment
	soul,				% strike through with \st{}
	textcomp,			% has \textuparrow
	xcolor,				% color!
%	xfrac,				% fancy fractions
}	

\definecolor{mygreen}{rgb}{0,0.6,0}
\definecolor{mygray}{rgb}{0.5,0.5,0.5}
\definecolor{mymauve}{rgb}{0.58,0,0.82}
\definecolor{darkblue}{rgb}{0,0,0.4}	

%% Graphics
%% --------------------
\usepackage{
	graphicx,			% allows insertion of images
	subfigure,			% allows subfigures (a), (b), etc.
	tikz,				% pretty pictures
}				
\graphicspath{ {graphics/} }	% (graphicx) relative path to graphics folder	
\usetikzlibrary{arrows, automata, calc, matrix, decorations.pathreplacing}			

%% Tables
%% --------------------------
\usepackage{
	booktabs,			% better tables, discourages vertical rulings
	multicol,			% allow multi columns
%			tocloft,			% finer control over TOC; enabled below due to subfigure conflict
}
%\usepackage[subfigure]{tocloft}
%\addtocontents{toc}{\cftpagenumbersoff{subsubsection}} % turn off subsubsection page numbers in ToC

%% Layout Alteration
%% --------------------------
\usepackage{			
%	caption,			% line breaks in captions with \\
%	changepage,			% change margins for PARTS of pages with (adjustwidth)
	fancyhdr,			% see config in LAYOUT AND STYLING
%	floatrow,			% multiple graphics, tables, etc in figure environment
	framed,				% nice boxes; used in Supervisor's Approval
%	fullpage,			% set full page margins
	geometry,			% change the margins for specific PAGES
%	lastpage,			% used with (fancyhdr)
	parskip,			% disable indents
	pdflscape,			% ???
	rotating,			% sideways figures
}
\geometry{						% specify page size options for (geometry)
	a4paper, 			% paper size
	hmargin=1in,		% horizontal margins
	vmargin=1in,		% vertical margins
}	


%% Units
%% --------------------------
\usepackage{
	siunitx,			% has S (decimal align) column type
}
\sisetup{input-symbols = {()},  % do not treat "(" and ")" in any special way
	group-digits  = false, 	% no grouping of digits
%	load-configurations = abbreviations,
%	per-mode = symbol,
}

%% Misc
%% --------------------------
\usepackage{
	enumitem,			% better control of enumerations, descriptions, etc
	listings,			% source code import and display
}

\lstset{ %
	language=verilog,				% the language of the code
	basicstyle=\footnotesize,       % the size of the fonts that are used for the code
	numbers=none,                   % where to put the line-numbers
	numberstyle=\tiny\color{mygray},% the style that is used for the line-numbers
	stepnumber=1,                   % the step between two line-numbers. If it's 1, each line
									% 	will be numbered
	numbersep=5pt,                  % how far the line-numbers are from the code
	backgroundcolor=\color{white},  % choose the background color. You must add \usepackage{color}
	showspaces=false,               % show spaces adding particular underscores
	showstringspaces=false,         % underline spaces within strings
	showtabs=false,                 % show tabs within strings adding particular underscores
	frame=single,	                % box the code [single, none]
	rulecolor=\color{black},        % if not set, the frame-color may be changed on line-breaks
									% 	within not-black text (e.g. commens (green here))
	tabsize=2,                      % sets default tabsize to 2 spaces
	captionpos=b,                   % sets the caption-position to bottom
	breaklines=true,                % sets automatic line breaking
	breakatwhitespace=false,        % sets if automatic breaks should only happen at whitespace
	title=\lstname,                 % show the filename of files included with \lstinputlisting;
									% 	also try caption instead of title
	keywordstyle=[1]\bfseries\color{darkblue},    % keyword style for mnemonics
	keywordstyle=[2]\bfseries\color{violet},	% keyword style for . mnemonics
	commentstyle=\color{mygreen},   % comment style
	stringstyle=\color{mymauve},    % string literal style
	escapeinside={\%*}{*)},         % if you want to add a comment within your code
	morekeywords={*,...}           	% if you want to add more keywords to the set
}

%----------------------------------------------------------------------------
%		     MACROS AND COMMANDS
%----------------------------------------------------------------------------

% Defines a new command for the horizontal lines, change thickness here
\newcommand{\HRule}{\rule{\linewidth}{0.5mm}} 

% override S column type with centered text column
\newcommand{\textcol}[1]{\multicolumn{1}{c}{#1}}


% Karnaugh maps! 
% http://tex.stackexchange.com/questions/140567/drawing-karnaughs-maps-in-latex
%isolated term
%#1 - Optional. Space between node and grouping line. Default=0
%#2 - node
%#3 - filling color
\newcommand{\implicantsol}[3][0]{
	\draw[rounded corners=3pt, fill=#3, opacity=0.3] ($(#2.north west)+(135:#1)$) rectangle ($(#2.south east)+(-45:#1)$);
}


%internal group
%#1 - Optional. Space between node and grouping line. Default=0
%#2 - top left node
%#3 - bottom right node
%#4 - filling color
\newcommand{\implicant}[4][0]{
	\draw[rounded corners=3pt, fill=#4, opacity=0.3] ($(#2.north west)+(135:#1)$) rectangle ($(#3.south east)+(-45:#1)$);
}

%group lateral borders
%#1 - Optional. Space between node and grouping line. Default=0
%#2 - top left node
%#3 - bottom right node
%#4 - filling color
\newcommand{\implicantlateral}[4][0]{
	\draw[rounded corners=3pt, fill=#4, opacity=0.3] ($(rf.east |- #2.north)+(90:#1)$)-| ($(#2.east)+(0:#1)$) |- ($(rf.east |- #3.south)+(-90:#1)$);
	\draw[rounded corners=3pt, fill=#4, opacity=0.3] ($(cf.west |- #2.north)+(90:#1)$) -| ($(#3.west)+(180:#1)$) |- ($(cf.west |- #3.south)+(-90:#1)$);
}

%group top-bottom borders
%#1 - Optional. Space between node and grouping line. Default=0
%#2 - top left node
%#3 - bottom right node
%#4 - filling color
\newcommand{\implicanttopbottom}[4][0]{
	\draw[rounded corners=3pt, fill=#4, opacity=0.3] ($(cf.south -| #2.west)+(180:#1)$) |- ($(#2.south)+(-90:#1)$) -| ($(cf.south -| #3.east)+(0:#1)$);
	\draw[rounded corners=3pt, fill=#4, opacity=0.3] ($(rf.north -| #2.west)+(180:#1)$) |- ($(#3.north)+(90:#1)$) -| ($(rf.north -| #3.east)+(0:#1)$);
}

%group corners
%#1 - Optional. Space between node and grouping line. Default=0
%#2 - filling color
\newcommand{\implicantcorners}[2][0]{
	\draw[rounded corners=3pt, opacity=.3] ($(rf.east |- 0.south)+(-90:#1)$) -| ($(0.east |- cf.south)+(0:#1)$);
	\draw[rounded corners=3pt, opacity=.3] ($(rf.east |- 8.north)+(90:#1)$) -| ($(8.east |- rf.north)+(0:#1)$);
	\draw[rounded corners=3pt, opacity=.3] ($(cf.west |- 2.south)+(-90:#1)$) -| ($(2.west |- cf.south)+(180:#1)$);
	\draw[rounded corners=3pt, opacity=.3] ($(cf.west |- 10.north)+(90:#1)$) -| ($(10.west |- rf.north)+(180:#1)$);
	\fill[rounded corners=3pt, fill=#2, opacity=.3] ($(rf.east |- 0.south)+(-90:#1)$) -|  ($(0.east |- cf.south)+(0:#1)$) [sharp corners] ($(rf.east |- 0.south)+(-90:#1)$) |-  ($(0.east |- cf.south)+(0:#1)$) ;
	\fill[rounded corners=3pt, fill=#2, opacity=.3] ($(rf.east |- 8.north)+(90:#1)$) -| ($(8.east |- rf.north)+(0:#1)$) [sharp corners] ($(rf.east |- 8.north)+(90:#1)$) |- ($(8.east |- rf.north)+(0:#1)$) ;
	\fill[rounded corners=3pt, fill=#2, opacity=.3] ($(cf.west |- 2.south)+(-90:#1)$) -| ($(2.west |- cf.south)+(180:#1)$) [sharp corners]($(cf.west |- 2.south)+(-90:#1)$) |- ($(2.west |- cf.south)+(180:#1)$) ;
	\fill[rounded corners=3pt, fill=#2, opacity=.3] ($(cf.west |- 10.north)+(90:#1)$) -| ($(10.west |- rf.north)+(180:#1)$) [sharp corners] ($(cf.west |- 10.north)+(90:#1)$) |- ($(10.west |- rf.north)+(180:#1)$) ;
}

%Empty Karnaugh map 4x4
\newenvironment{Karnaugh4x4}[2]%
{
	\begin{tikzpicture}[baseline=(current bounding box.north),scale=0.8]
	\draw (0,0) grid (4,4);
	\draw (0,4) -- node [pos=0.7,above right,anchor=south west] {#2} node [pos=0.7,below left,anchor=north east] {#1} ++(135:1);
	%
	\matrix (mapa) [matrix of nodes,
	column sep={0.8cm,between origins},
	row sep={0.8cm,between origins},
	every node/.style={minimum size=0.3mm},
	anchor=8.center,
	ampersand replacement=\&] at (0.5,0.5)
	{
		\& |(c00)| 00         \& |(c01)| 01         \& |(c11)| 11         \& |(c10)| 10         \& |(cf)| \phantom{00} \\
		|(r00)| 00             \& |(0)|  \phantom{0} \& |(1)|  \phantom{0} \& |(3)|  \phantom{0} \& |(2)|  \phantom{0} \&                     \\
		|(r01)| 01             \& |(4)|  \phantom{0} \& |(5)|  \phantom{0} \& |(7)|  \phantom{0} \& |(6)|  \phantom{0} \&                     \\
		|(r11)| 11             \& |(12)| \phantom{0} \& |(13)| \phantom{0} \& |(15)| \phantom{0} \& |(14)| \phantom{0} \&                     \\
		|(r10)| 10             \& |(8)|  \phantom{0} \& |(9)|  \phantom{0} \& |(11)| \phantom{0} \& |(10)| \phantom{0} \&                     \\
		|(rf) | \phantom{00}   \&                    \&                    \&                    \&                    \&                     \\
	};
}%
{
	\end{tikzpicture}
}

%Empty Karnaugh map 2x4
\newenvironment{Karnaugh2x4}[2]%
{
	\begin{tikzpicture}[baseline=(current bounding box.north),scale=0.8]
	\draw (0,0) grid (4,2);
	\draw (0,2) -- node [pos=0.7,above right,anchor=south west] {#2} node [pos=0.7,below left,anchor=north east] {#1} ++(135:1);
	%
	\matrix (mapa) [matrix of nodes,
	column sep={0.8cm,between origins},
	row sep={0.8cm,between origins},
	every node/.style={minimum size=0.3mm},
	anchor=4.center,
	ampersand replacement=\&] at (0.5,0.5)
	{
		\& |(c00)| 00         \& |(c01)| 01         \& |(c11)| 11         \& |(c10)| 10         \& |(cf)| \phantom{00} \\
		|(r00)| 0             \& |(0)|  \phantom{0} \& |(1)|  \phantom{0} \& |(3)|  \phantom{0} \& |(2)|  \phantom{0} \&                     \\
		|(r01)| 1             \& |(4)|  \phantom{0} \& |(5)|  \phantom{0} \& |(7)|  \phantom{0} \& |(6)|  \phantom{0} \&                     \\
		|(rf) | \phantom{00}  \&                    \&                    \&                    \&                    \&                     \\
	};
}%
{
	\end{tikzpicture}
}

%Empty Karnaugh map 2x2
\newenvironment{Karnaugh2x2}[2]%
{
	\begin{tikzpicture}[baseline=(current bounding box.north),scale=0.8]
	\draw (0,0) grid (2,2);
	\draw (0,2) -- node [pos=0.7,above right,anchor=south west] {#2} node [pos=0.7,below left,anchor=north east] {#1} ++(135:1);
	%
	\matrix (mapa) [matrix of nodes,
	column sep={0.8cm,between origins},
	row sep={0.8cm,between origins},
	every node/.style={minimum size=0.3mm},
	anchor=2.center,
	ampersand replacement=\&] at (0.5,0.5)
	{
		\& |(c00)| 0          \& |(c01)| 1  \\
		|(r00)| 0 \& |(0)|  \phantom{0} \& |(1)|  \phantom{0} \\
		|(r01)| 1 \& |(2)|  \phantom{0} \& |(3)|  \phantom{0} \\
	};
}%
{
	\end{tikzpicture}
}

%Defines 8 or 16 values (0,1,X)
\newcommand{\contingut}[1]{%
	\foreach \x [count=\xi from 0]  in {#1}
	\path (\xi) node {\x};
}

%Places 1 in listed positions
\newcommand{\minterms}[1]{%
	\foreach \x in {#1}
	\path (\x) node {1};
}

%Places 0 in listed positions
\newcommand{\maxterms}[1]{%
	\foreach \x in {#1}
	\path (\x) node {0};
}

%Places X in listed positions
\newcommand{\dontcare}[1]{%
	\foreach \x in {#1}
	\path (\x) node {X};
}


%----------------------------------------------------------------------------
%----------------------------------------------------------------------------
%				   DOCUMENT
%----------------------------------------------------------------------------
%----------------------------------------------------------------------------

\begin{document}

%----------------------------------------------------------------------------
%				    TITLE PAGE
%----------------------------------------------------------------------------

\begin{titlepage}

\center
 
% Header
\textsc{\LARGE University of Victoria}\\[1cm] 	% Name of your university/college
\textsc{\Large CENG 241}\\[0.5cm] 			% Major heading such as course name
\textsc{\large Digital Design I}\\[0.5cm] 		% Minor heading such as course title


% Lab Title
\HRule \\[0.4cm]
{\huge \bfseries Lab 7:  RAM System}\\[0.2cm] % Title of your document
\HRule \\[1.5cm]
 
 
%Lab Instructor Details
\begin{minipage}{0.7\textwidth}
\begin{flushleft} 

\large\emph{Instructor:} \\
Dr. Amirali \textsc{Baniasadi} \\
\vspace{12 pt}
\emph{Teaching Assistant:} \\
Grace \textsc{Hui}

\end{flushleft}
\end{minipage}
~
%% No content here, but it keeps the alignment of the instructor/TA
%% box correct.
%% Consider revising.
\begin{minipage}{0.1\textwidth}
\begin{flushright} \large
%Dr. Barbara \textsc{Sawicka} \\
\vspace{12 pt}
%\emph{Teaching Assistant:} \\
%Vahid \textsc{Moradi}
\end{flushright}
\end{minipage}\\[2cm]


% Lab members
\Large Yves \textsc{S\'{e}n\'{e}chal}
\large V00213837	\\
\Large Tyler \textsc{Stephen}
\large V00812021	\\
A01 - B03\\[1.5cm] 


% Date
{\large July 27, 2015}\\ % Date, change the \today to a set date if you want to be precise

% Logo
\begin{figure}[b]	 % put logo at bottom of the page
	\centering
	\includegraphics[scale=0.3]{UVic_logo}
\end{figure}

\end{titlepage}

%----------------------------------------------------------------------------
%				    BODY
%----------------------------------------------------------------------------

\section{Introduction}

The RAM controller reads and writes data to RAM modules. In this lab, the RAM controller was designed and implemented using D-FFs in a system similar to figure \ref{fig:controller}. 

\section{Discussion}\label{sec:discussion}

The memory system built is shown in figure \ref {fig:controller}. The state diagram, included on page 3, determines the proper sequence to set the RAM controller into read or write mode. The RAM write operation cycle is summarized by the following: 

\begin{itemize}
 
 	 \item the RAM is set to accept data,
	 \item the 4-bit counter generates an address for the data,
 	 \item the buffer is active and transmits its data to the RAM at the specified address,
	 \item the register is in stand-by state.
	  
\end{itemize}

The RAM read cycle is summarized by the following:

\begin{itemize}
 
 	 \item the RAM is set to output data,
	 \item the value of the 4-bit counter is the location of the read data,
 	 \item the buffer is set to high impedance mode which blocks all data to the RAM,
	 \item the register is active and outputs the data received from the RAM.
	  
\end{itemize}

\begin{figure}[htpb]
	\centering
	\includegraphics[scale=0.5]{controller}
	\caption{RAM controller and memory system}
	\label{fig:controller}
\end{figure}

\subsection{Timing considerations}

The read and write timing diagrams, shown in figure \ref{fig:read-timing} and figure \ref{fig:write-timing} respectively, indicate that the moment in which the data is valid is offset from moment where the address is valid; the data is valid after the address is valid. The access time and hold time for the read operation and write operation respectively can be attributed to the RAM's internal delay of \SI{85}{\nano\second} maximum. However, this does not pose a problem with a manual clock, but could become a nuisance should high-frequency clock be used.

\begin{figure}[htpb]
	\centering
	\includegraphics[scale=0.5]{read}
	\caption{Timing diagram for the RAM read operation}
	\label{fig:read-timing}
\end{figure}

\begin{figure}[htpb]
	\centering
	\includegraphics[scale=0.5]{write}
	\caption{Timing diagram for the RAM write operation}
	\label{fig:write-timing}
\end{figure}

\section{RAM controller design process}
\subsection{State diagram}

The state diagram of the RAM controller is shown below. The system was designed as a Moore machine with the transition tables displayed in figure \ref{table:transition-moore}. 

\begin{figure}[htpb]
	\centering
	\begin{tikzpicture}
	[>=stealth', shorten >= 1pt, auto, node distance=4cm]
	\node[state, initial]	(a) []					{$\overline{Read}=0$};
	\node[state, align=left](b) [above left of=a]	{$\overline{OE}=0$\\$\overline{WE}=1$};
	\node[state]			(c) [above right of=b]	{$Load=1$};
	\node[state]			(d) [right of=a]	{$Incr=1$};
	\node[state, align=left](e) [below left of=a]	{$\overline{OE}=0$\\$\overline{WE}=0$};
	\node[state, align=center](f) [below right of=e]{Load into \\ RAM};
	
	\path[->]
	(a) edge [bend left]	node [right, align=left, pos=0.7, yshift=5]	{$Go=1$\\$R/\overline{W}=1$}		(b)
	edge [loop above]	node [above]	{$Go=0$}		(a)
	(b)	edge [bend left]	node			{\textuparrow}	(c)
	(c)	edge [bend left]	node			{\textuparrow}	(d)
	(d) edge 				node [right, yshift=15]{\textuparrow}	(a)
	(a) edge [bend right]	node [right, align=left, pos=0.7, yshift=-5]	{$Go=1$\\$R/\overline{W}=0$}		(e)
	(e)	edge [bend right]	node [left, yshift=-10]{\textuparrow}	(f)
	(f)	edge [bend right]	node [right, yshift=-10]{\textuparrow}	(d)
	;
	
	\draw [decorate,decoration={brace,amplitude=10pt,mirror},xshift=-4pt,yshift=0pt]
	(6.5,1.0) -- (6.5,6.5) node [black,midway,xshift=2.5cm,align=left] 
	{Read from \\ RAM};
	
	%		\draw [decorate,decoration={brace,amplitude=10pt},xshift=-4pt,yshift=0pt]
	%		(10.5,0.95) -- (10.5,-0.95) node [black,midway,xshift=0.5cm,align=left] 
	%		{Steady \\ state};
	
	\draw [decorate,decoration={brace,amplitude=10pt},xshift=-4pt,yshift=0pt]
	(6.5,-1.0) -- (6.5,-6.5) node [black,midway,xshift=0.5cm,align=left] 
	{Write to \\ RAM};
	\end{tikzpicture}
\end{figure}	

\newpage

\begin{figure}[htpb]
	\centering
	\subfigure[State enumeration]
	{
		\begin{tabular}{c | c c c }
			State & $S_2$ & $S_1$ & $S_0$ \\
			\hline
			$a$ & 0 & 0 & 0 \\
			$b$ & 0 & 0 & 1 \\
			$c$ & 0 & 1 & 0 \\
			$d$ & 0 & 1 & 1 \\
			$e$ & 1 & 0 & 0 \\
			$f$ & 1 & 0 & 1 \\
			 -  & 1 & 1 & 0 \\
			 -  & 1 & 1 & 1 \\
		\end{tabular}
	}
	\subfigure[Next state]
	{
		\begin{tabular}{c c c c c | c c c}
			$S_2$ & $S_1$ & $S_0$ & $Go$ & $R/\overline{W}$ & $S_2^+$ & $S_1^+$ & $S_0^+$ \\
			\hline
			0 & 0 & 0 & 0 & 0 & 0 & 0 & 0 \\
			0 & 0 & 0 & 0 & 1 & 0 & 0 & 0 \\
			0 & 0 & 0 & 1 & 0 & 1 & 0 & 0 \\
			0 & 0 & 0 & 1 & 1 & 0 & 0 & 1 \\
			0 & 0 & 1 & 0 & 0 & 0 & 1 & 0 \\
			0 & 0 & 1 & 0 & 1 & 0 & 1 & 0 \\
			0 & 0 & 1 & 1 & 0 & 0 & 1 & 0 \\
			0 & 0 & 1 & 1 & 1 & 0 & 1 & 0 \\
			0 & 1 & 0 & 0 & 0 & 0 & 1 & 1 \\
			0 & 1 & 0 & 0 & 1 & 0 & 1 & 1 \\
			0 & 1 & 0 & 1 & 0 & 0 & 1 & 1 \\
			0 & 1 & 0 & 1 & 1 & 0 & 1 & 1 \\
			0 & 1 & 1 & 0 & 0 & 0 & 0 & 0 \\
			0 & 1 & 1 & 0 & 1 & 0 & 0 & 0 \\
			0 & 1 & 1 & 1 & 0 & 0 & 0 & 0 \\
			0 & 1 & 1 & 1 & 1 & 0 & 0 & 0 \\
			1 & 0 & 0 & 0 & 0 & 1 & 0 & 1 \\
			1 & 0 & 0 & 0 & 1 & 1 & 0 & 1 \\
			1 & 0 & 0 & 1 & 0 & 1 & 0 & 1 \\
			1 & 0 & 0 & 1 & 1 & 1 & 0 & 1 \\
			1 & 0 & 1 & 0 & 0 & 0 & 1 & 1 \\
			1 & 0 & 1 & 0 & 1 & 0 & 1 & 1 \\
			1 & 0 & 1 & 1 & 0 & 0 & 1 & 1 \\
			1 & 0 & 1 & 1 & 1 & 0 & 1 & 1 \\
			1 & 1 & 0 & 0 & 0 & - & - & - \\
			1 & 1 & 0 & 0 & 1 & - & - & - \\
			1 & 1 & 0 & 1 & 0 & - & - & - \\
			1 & 1 & 0 & 1 & 1 & - & - & - \\
			1 & 1 & 1 & 0 & 0 & - & - & - \\
			1 & 1 & 1 & 0 & 1 & - & - & - \\
			1 & 1 & 1 & 1 & 0 & - & - & - \\
			1 & 1 & 1 & 1 & 1 & - & - & - \\
		\end{tabular}
	}
	\caption[figurename=Table]{Transition tables for the Moore machine}
	\label{table:transition-moore}
\end{figure}


From here three karnaugh maps were used to resolve the next states $S_2^+$, $S_1^+$, and $S_0^+$. Their state equations are shown below.

\begin{align*}
	S_2^+ &= S_2 S_0^{\prime} + S_1^{\prime} S_0^{\prime}\; Go\; R/\overline{W}\\
	S_1^+ &= S_1\; \text{XOR}\; S_0 \\
	S_0^+ &= S_2 + S_1 S_0^{\prime} + S_0^{\prime}\; Go\; R/\overline{W} \\
\end{align*}

\newpage

The state outputs are summarized in table \ref{table:state_output}. Five karnaugh maps were needed to resolve their state equations which are shown below.

\begin{align*}
	\overline{OE} &= S_1^{\prime} S_0^{\prime} + S_1 S_0 \\
	\overline{WE} &= S_2^{\prime} \\
	Load &= S_1 S_0^{\prime} \\
	\overline{Read} &= S_1 + S_2^{\prime} S_0 \\
	Incr &= S_1 S_0 \\
\end{align*}

\begin{table}
	\centering
		\begin{tabular}{c c c | c c c c c }
			$S_2$ & $S_1$ & $S_0$ & $\overline{OE}$ & $\overline{WE}$ & $Load$ & $\overline{Read}$ & $Incr$\\
			\hline
			0 & 0 & 0 & 1 & 1 & 0 & 0 & 0 \\
			0 & 0 & 1 & 0 & 1 & 0 & 1 & 0 \\
			0 & 1 & 0 & 0 & 1 & 1 & 1 & 0 \\
			0 & 1 & 1 & 1 & 1 & 0 & X & 1 \\
			1 & 0 & 0 & X & 0 & 0 & 0 & 0 \\
			1 & 0 & 1 & X & 0 & 0 & 0 & 0 \\
			1 & 1 & 0 & - & - & - & - & - \\
			1 & 1 & 1 & - & - & - & - & - \\
		\end{tabular}
	\caption{State output}
	\label{table:state_output}
\end{table}

\subsection{Controller logic}

The schematic of the logic design for the RAM controller is included in figure \ref{fig:schematic}. This design was implemented using three D-FFs plus some external logic.

\begin{figure}[htpb]
	\centering
	\includegraphics[scale=0.5]{controller.pdf}
	\caption{RAM controller logic design}
	\label{fig:schematic}
\end{figure}


\section{Conclusion}

A functioning 4-bit RAM controller was built using D-FFs and some external logic. The system had a 4-bit counter which determined the address of the data. Four LEDs displayed the present address, while, when in read mode, four other LEDs revealed the stored data at that address. 

\end{document}